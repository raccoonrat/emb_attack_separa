\documentclass[10pt,twocolumn,letterpaper]{ctexart}

\usepackage{usenix2020_SOUPS}
\usepackage{amsmath}
\usepackage{amssymb}
\usepackage{amsthm}
\usepackage{graphicx}
\usepackage{url}
\usepackage{hyperref}
\usepackage{xcolor}

% 设置中文字体(如果需要)
% \setCJKmainfont{SimSun}
% \setCJKsansfont{SimHei}
% \setCJKmonofont{FangSong}

% Theorem environments
\newtheorem{theorem}{定理}
\newtheorem{lemma}{引理}
\newtheorem{corollary}{推论}
\newtheorem{definition}{定义}
\newtheorem{proposition}{命题}

% Custom commands
\newcommand{\KL}[2]{D_{\text{KL}}(#1 \| #2)}
\newcommand{\TV}[2]{\| #1 - #2 \|_{\text{TV}}}
\newcommand{\Dstar}{D^*}
\newcommand{\Dstaradv}{D^*_{\text{adv}}}

\title{水印技术的范式转变:MoE专家激活水印对抗释义攻击的理论优势}

% 定义USENIX模板需要的命令(如果模板中没有定义)
\providecommand{\alignauthor}{}
\providecommand{\affaddr}[1]{#1}
\providecommand{\email}[1]{\texttt{#1}}

\author{
\alignauthor
跨学科研究团队\\
\affaddr{跨学科研究院}\\
\email{anonymous@example.edu}
}

\begin{document}

\maketitle

\begin{abstract}
大语言模型(LLM)已经彻底改变了自然语言处理领域,但其广泛部署引发了关于内容归属和版权保护的关键担忧。水印技术已成为一种有前景的解决方案,然而现有的基于词元-逻辑值(token-logit)的水印方案在面对释义攻击时存在根本性脆弱性。本文提出了一种基于混合专家模型(MoE)架构的新型水印范式的理论分析,通过信息论原理证明了其固有的鲁棒性优势。我们证明了传统稠密模型水印在释义攻击下检测能力呈线性衰减($O(\gamma)$),而基于MoE的水印实现了次线性衰减($O(\sqrt{\gamma})$),为改进对抗鲁棒性提供了数学上严格的基础。这种优势背后的关键机制是\textit{信号-攻击解耦}原理:水印信号嵌入在内部专家激活空间中,而攻击操作在外部词元空间中。我们的分析建立了检测能力退化的形式化边界,并提供了可证明的水印鲁棒性框架。实验验证标记为待完成,详细证明将在后续工作中补充。
\end{abstract}

\section{引言}

\subsection{水印的必要性与对抗脆弱性困境}

大语言模型(LLM)的水印技术已成为安全性和所有权验证中的关键问题。作为一种在模型生成内容中嵌入隐藏模式的技术,水印被认为是区分人类与AI生成内容、追踪内容分发以及应对版权挑战的有效手段。

然而,尽管水印嵌入机制日益成熟,其\textit{对抗脆弱性}仍然是实际部署的核心障碍。现有的水印方法,特别是那些应用于稠密模型的方法,极易受到\textit{擦洗攻击}~\cite{scrubbing}的破坏。这类攻击,特别是通过多轮释义实现的攻击,已被证明是破坏水印的高效手段。攻击者可以通过同义词替换、句法重组、跨语言回译或摘要,在保持文本核心语义的同时,系统性地破坏水印所依赖的统计信号。

压倒性的研究证据表明,当前主流的基于词元的水印技术是脆弱的~\cite{token_fragile}。这种脆弱性暴露了现有水印范式的根本缺陷:水印信号的生存能力远低于其嵌入能力。因此,迫切需要一种新的水印范式,它不仅能嵌入信号,还能确保信号在经历语义保持变换(即释义攻击)后依然可被检测。

\subsection{核心论点:两种范式的机理分野}

本文旨在从信息论和统计检验的理论高度,推导一种基于混合专家模型(MoE)架构的新型水印范式,并证明其在面对释义攻击时,为何在检测能力衰减机理上(而非仅仅是经验表现上)优于传统稠密模型水印范式。

\textbf{范式A(稠密模型):信号-攻击重合}

我们首先解构当前流行的、以Kirchenbauer等人~\cite{kirchenbauer2023}为代表的词元-逻辑值水印方案。我们将从机理上论证,其根本弱点在于\textit{信号-攻击重合}:水印信号(即被偏置的词元选择概率)与其所对抗的攻击(即词汇替换)存在于完全相同的向量空间——词汇表空间。

\textbf{范式B(MoE模型):信号-攻击解耦}

随后,基于核心研究~\cite{moe_watermark}中提出的理论框架,我们推导一种全新的MoE水印范式。该范式利用了MoE架构的独特性(即稀疏激活)~\cite{moe_arch}。我们将从理论上证明,其鲁棒性来源于\textit{信号-攻击解耦}:水印信号被嵌入在模型的内部专家激活模式$g(x)$这一隐写空间中~\cite{moe_watermark},而攻击者只能在外部词元空间($x \rightarrow x'$)中操作。

\textbf{核心机理推导(本文目标)}

本文的最终目标是证明,这种\textit{解耦}在数学上如何体现为一个可量化的、根本性的优势。我们将推导出,在面对释义攻击(被建模为输入分布偏移$\gamma$)时,范式A的检测能力(z-score)呈线性衰减($O(\gamma)$),而范式B的检测能力(Chernoff信息$\Dstar$)则呈次线性衰减($O(\sqrt{\gamma})$)~\cite{moe_watermark}。这种次线性衰减,是MoE水印范式在对抗释义攻击时具有机理优势的严格数学证明。

\section{范式A:Kirchenbauer水印及其信号-攻击重合困境}
\label{sec:paradigm_a}

为了理解MoE范式的优越性,我们必须首先严格解构作为基线的稠密模型水印范式(下文简称KGW范式)。

\subsection{机制解构:``绿名单''偏置}

KGW范式的核心机制是在文本生成过程中的采样阶段对词元的逻辑值(logits)进行偏置~\cite{kirchenbauer2023}。

\begin{enumerate}
\item \textbf{分区}:在生成每个词元之前,该方法使用一个伪随机种子(通常是前一个或前$k$个词元)将整个词汇表$\mathcal{V}$划分为两个子集:``绿名单''$G$(占比$\gamma$)和``红名单''$R$(占比$1-\gamma$)~\cite{kirchenbauer2023}。
\item \textbf{偏置}:随后,一个恒定的正向偏置$\delta$被施加到所有``绿名单''中词元的原始逻辑值上~\cite{kirchenbauer2023}。
\item \textbf{采样}:模型从这个被偏置(或称``扭曲'')的概率分布中采样下一个词元~\cite{kirchenbauer2023}。
\end{enumerate}

这种机制的直接后果是,水印模型在统计上会更频繁地选择``绿名单''中的词元。水印信号被完全编码在最终输出的\textbf{词元概率分布}中~\cite{kirchenbauer2023}。

\subsection{检测原理:z-score与词元频率的脆弱性}

KGW范式的检测原理与其嵌入机制相对应,它依赖于对词元频率的统计检验~\cite{kirchenbauer2023}。

检测器(知晓用于分区的伪随机种子)会遍历待检测文本,计算落在``绿名单''中的词元数量$k$。在零假设($H_0$,即文本非水印生成)下,$k$的期望值应接近$N \cdot \gamma$($N$为文本总长度)。在备择假设($H_1$,即文本为水印生成)下,由于$\delta$偏置的存在,$k$的值将显著高于$N \cdot \gamma$。

这种显著性是通过z-score统计量来衡量的~\cite{kirchenbauer2023}。一个足够高(例如$>4.0$)的z-score被认为是水印存在的有力证据。

然而,这种检测机制存在一个内在的困境:z-score的期望值(即信号强度)与水印偏置$\delta$直接相关,但$\delta$的增大会不可避免地扭曲原始语言模型的概率分布,导致生成文本的质量(通常以困惑度PPL衡量)下降~\cite{kirchenbauer2023}。水印设计者必须在``可检测性''(高$\delta$)和``隐蔽性''(低$\delta$,低PPL)之间做出妥协。

\subsection{释义攻击的灾难性影响:线性衰减}

KGW范式的真正崩溃点在于其\textit{信号-攻击重合}的脆弱性。

\begin{enumerate}
\item \textbf{信号载体}:水印信号的载体是\textbf{词元}(具体来说,是``绿名单''词元的\textit{出现频率})。
\item \textbf{攻击向量}:释义攻击,无论是同义词替换~\cite{paraphrase}、词汇编辑~\cite{lexical_edit}还是更复杂的跨语言摘要~\cite{cross_lang},其\textit{操作对象}同样是\textbf{词元}。
\end{enumerate}

当攻击向量与其试图破坏的信号载体完全重合时,攻击的效率是灾难性的。

\subsubsection{线性衰减的机理推导}

我们可以对这种衰减进行建模:

\begin{enumerate}
\item KGW范式下的信号强度(z-score)是``绿名单''词元数量$k$的函数。
\item 释义攻击$A$是一种进行词元替换、删除或重排的操作。
\item 假设一次释义攻击的``强度''$\gamma$被定义为``被编辑或替换的词元占总词元数的比例''(例如,编辑距离$L$)。
\item 当攻击者替换一个``绿名单''词元$t_g$时,他们有很大概率(例如,$(1-\gamma)$)会将其替换为一个``红名单''词元$t_r$。
\item 因此,z-score信号的损失$\Delta Z$与被替换的``绿名单''词元数量成正比,而后者又与攻击强度$\gamma$成正比。
\end{enumerate}

\begin{equation}
\Delta Z \propto \Delta k \propto \gamma
\end{equation}

这种关系被称为\textbf{线性衰减}。如果攻击者将10\%的词元($\gamma=0.1$)进行释义,他们将直接破坏约10\%的水印信号。如果他们将30\%的词元($\gamma=0.3$)进行释义~\cite{paraphrase},他们就会破坏约30\%的信号。这种脆弱的线性关系意味着,即使是中等强度的释义攻击,也能轻易地将z-score压低到检测阈值以下,使水印失效。

表~\ref{tab:paradigm_a}总结了KGW范式A的机理及其核心困境。

\begin{table}[h]
\centering
\caption{稠密模型(KGW)水印范式分析(范式A)}
\label{tab:paradigm_a}
\begin{tabular}{|l|l|}
\hline
\textbf{属性} & \textbf{范式A:KGW稠密模型水印} \\
\hline
信号载体 & 词元逻辑值;词汇表空间~\cite{kirchenbauer2023} \\
嵌入机制 & 逻辑值偏置$\delta$(``绿名单'')~\cite{kirchenbauer2023} \\
检测理论 & 词元频率计数(z检验)~\cite{kirchenbauer2023} \\
攻击向量 & 词汇替换;释义~\cite{paraphrase} \\
核心漏洞 & \textbf{信号-攻击重合} \\
衰减机理 & \textbf{线性衰减}:$Decay \propto O(\gamma)$ \\
\hline
\end{tabular}
\end{table}

\section{范式B:MoE模型——基于信息论的假设检验新范式}
\label{sec:paradigm_b}

面对范式A的``线性衰减''困境,MoE水印范式~\cite{moe_watermark}采用了根本不同的设计哲学。通过利用MoE架构的独特性,它将水印信号从脆弱的``词元空间''转移到隐蔽的``激活空间'',从而实现了\textit{信号-攻击解耦}。

\subsection{信号-攻击解耦:新的隐写信道}

MoE架构用稀疏的MoE层取代了传统Transformer中的密集前馈网络(FFN)层~\cite{moe_arch}。在每次前向传播中,一个``门控网络''$g(x)$会根据当前输入$x$动态地从$K$个总专家中选择一个小的子集(例如,Top-$k$,$S=2$)来激活~\cite{moe_watermark}。

MoE范式的核心洞察在于:这个\textbf{动态的、随输入变化的专家激活模式$g(x)$}本身,可以被用作一个全新的、高带宽的\textbf{隐写信道}~\cite{moe_watermark}。

这种设计立即实现了\textit{信号-攻击解耦}:

\begin{itemize}
\item \textbf{信号空间}:水印被嵌入在模型的\textbf{内部专家激活分布}$p(e|x)$中。
\item \textbf{攻击空间}:释义攻击者只能观察和修改\textbf{外部的输入/输出词元}($x \rightarrow x'$,$y \rightarrow y'$)。
\end{itemize}

攻击者无法直接观测或操纵$g(x)$的选择。他们对$x$的修改(释义)会\textit{间接}影响$g(x)$,但这种影响不再是范式A中那种``一一对应''的直接破坏。

\subsection{范式转变:从``频率计数''到``假设检验''}

KGW范式(第~\ref{sec:paradigm_a}节)使用z-score进行频率计数,这在统计上是次优的。MoE范式~\cite{moe_watermark}则从根本上将水印检测问题建模为一个严格的\textbf{二元假设检验}问题~\cite{moe_watermark}。

我们定义两个关于专家激活模式$S_i = g(x_i)$的概率分布:

\begin{itemize}
\item \textbf{零假设$H_0$(模型无水印)}:激活模式遵循原始的、未经修改的门控分布$p_0(e|x)$。
\item \textbf{备择假设$H_1$(模型有水印)}:激活模式遵循被轻微修改过的水印分布$p_1(e|x)$。
\end{itemize}

水印的嵌入过程(见~\cite{moe_watermark})即是通过修改门控网络的逻辑值(logits),将$p_0$变为$p_1$。这个修改过程受到一个严格的KL散度约束:$\KL{p_1}{p_0} \le \epsilon$。这个$\epsilon$非常小,它既是水印的``强度'',也保证了水印的``隐蔽性'',即对模型原始性能(如精度)的影响最小化~\cite{moe_watermark}。

\subsection{最优检测机制:Neyman-Pearson引理}

一旦问题被形式化为$H_0$ vs. $H_1$的假设检验,\textbf{Neyman-Pearson引理}~\cite{moe_watermark}便给出了该问题的\textbf{最优检测器}。

该引理指出,在给定假正率(Type I Error)上界$\alpha$的情况下,能够最大化检测能力(即最小化假负率Type II Error)的最强大的检验是\textbf{似然比检验(LLR)}~\cite{moe_watermark}。

具体而言,我们观测$n$次推理($n$个样本)所对应的专家激活模式$X_1, \ldots, X_n$。我们计算这组观测在$H_1$和$H_0$下的对数似然比(LLR):

\begin{equation}
\Lambda_n = \sum_{i=1}^n \log \frac{p_1(X_i)}{p_0(X_i)}
\end{equation}

然后,我们将$\Lambda_n$与一个由$\alpha$决定的阈值$\tau_\alpha$进行比较:

\begin{itemize}
\item 若$\Lambda_n > \tau_\alpha$,则判为$H_1$(有水印)。
\item 若$\Lambda_n \le \tau_\alpha$,则判为$H_0$(无水印)。
\end{itemize}

这是从z-score频率计数到信息论最优检测的深刻转变。检测器不再是简单地``计数'',而是计算观测到的``证据序列''($X_1, \ldots, X_n$)在两个``世界模型''($p_0$和$p_1$)下的相对可信度。

\subsection{鲁棒性的核心度量:Chernoff信息($\Dstar$)}
\label{sec:chernoff}

最优检测器(LLR)的引入,自然地导出了一个衡量鲁棒性的核心度量。\textbf{Chernoff-Stein定理}~\cite{moe_watermark}描述了LLR检验的错误率$P_e$如何随样本数$n$渐近衰减。

该定理指出,错误率$P_e$呈\textbf{指数级}衰减:

\begin{equation}
\log P_e \sim -n \cdot \Dstar(p_0, p_1)
\end{equation}

这里的$\Dstar(p_0, p_1)$就是\textbf{Chernoff信息}。它是衡量$p_0$和$p_1$这两个分布``可区分度''的核心信息论度量~\cite{moe_watermark}。

$\Dstar$的物理意义是:

\begin{itemize}
\item $\Dstar$越大,$p_0$和$p_1$的差异越大,两者越容易区分,错误率$P_e$衰减越快。
\item $\Dstar$越小,两者越接近,越难区分,错误率$P_e$衰减越慢。
\end{itemize}

进而,推论3.1~\cite{moe_watermark}给出,要达到某个目标检测精度(例如$\delta=0.01$,即99\%准确率),所需的样本数$n^*$与$\Dstar$成反比:

\begin{equation}
n^* \approx \frac{\log(1/\delta)}{\Dstar}
\end{equation}

例如,要达到99\%的检测准确率,如果$\Dstar=0.1$,需要约46个样本;如果$\Dstar=0.05$,则需要约92个样本~\cite{moe_watermark}。

这一理论转变至关重要。它将``水印检测能力''这个问题,从一个模糊的``z-score'',提炼为了一个精确的信息论度量$\Dstar$。因此,\textbf{所有关于水印鲁棒性与衰减的讨论,都精确地转化为一个问题:在对抗性攻击下,$\Dstar$是如何衰减的?}

\section{核心推导:MoE水印对抗衰减的次线性边界}
\label{sec:core_derivation}

现在我们已经建立了两个关键点:

\begin{enumerate}
\item 范式A(KGW)的检测能力(z-score)在攻击下呈\textbf{线性衰减}。
\item 范式B(MoE)的检测能力由$\Dstar$(Chernoff信息)衡量。
\end{enumerate}

本部分将推导范式B的核心优势:其$\Dstar$在攻击下呈\textbf{次线性衰减}。

\subsection{释义攻击的重新建模:输入分布偏移$\gamma$}

KGW范式(第~\ref{sec:paradigm_a}节)之所以脆弱,部分原因在于攻击(编辑距离$L$)与信号(词元计数$k$)之间的关系是混乱且难以建模的。

MoE范式~\cite{moe_watermark}则采用了一种更根本的、信息论的建模方式。它不关心攻击的具体形式(是同义词替换还是句法重组),而是对攻击的\textbf{效果}进行建模。

\textbf{释义攻击$x \rightarrow x'$的效果,被建模为对原始输入分布$D(X)$的扰动,使其变为一个新的分布$D(X')$}~\cite{moe_watermark}。

而这次攻击的\textbf{强度},则被严格地量化为这两个输入分布之间的KL散度:

\begin{equation}
\gamma = \KL{D(X')}{D(X)}
\label{eq:attack_strength}
\end{equation}

这是一个极其精妙的理论抽象。它将``释义''这一模糊的语言学概念,转化为了一个精确的信息论度量$\gamma$。$\gamma$(例如0.003 nats)~\cite{moe_watermark}衡量了释义攻击者对其输入数据流注入了多少``信息''或``畸变''。

\subsection{定理5.1深入分析:鲁棒性下界}

现在,我们的问题变得非常清晰:

\begin{itemize}
\item \textbf{原始检测能力}:$\Dstar = \Dstar(p_0, p_1)$
\item \textbf{攻击强度}:$\gamma = \KL{D(X')}{D(X)}$
\item \textbf{待求}:攻击后的新检测能力$\Dstaradv = \Dstar(p'_0, p'_1)$,其中$p'_0, p'_1$是受$\gamma$攻击扰动后的新激活分布。
\end{itemize}

\textbf{定理5.1(对抗鲁棒性)}~\cite{moe_watermark}给出了$\Dstaradv$的理论下界:

\begin{theorem}[对抗鲁棒性]
\label{thm:robustness}
在强度为$\gamma = \KL{D(X')}{D(X)}$的释义攻击下,对抗Chernoff信息$\Dstaradv$满足:
\begin{equation}
\Dstaradv \geq \Dstar(p_0, p_1) - C\sqrt{\gamma \cdot \Dstar(p_0, p_1)} - O(\gamma)
\label{eq:robustness_bound}
\end{equation}
其中$C$是一个与Pinsker不等式相关的常数(约1--2)~\cite{moe_watermark}。
\end{theorem}

\textit{证明:待后续补充。}

\subsection{机理阐释:为什么是$\sqrt{\gamma}$?}

用户查询的核心(``机理上的优势'')的答案就在这个$\sqrt{\gamma}$项中。这个平方根项不是凭空出现的,它是``信号-攻击解耦''在信息论基本定律下的必然数学结果。

\subsubsection{$\sqrt{\gamma}$衰减的机理推导链:}

\begin{enumerate}
\item \textbf{空间解耦}:再次明确,攻击强度$\gamma$存在于\textbf{输入空间}($D(X) \rightarrow D(X')$),而信号$\Dstar$存在于\textbf{专家激活空间}($p_i \rightarrow p'_i$)。

\item \textbf{攻击的传播}:攻击$\gamma$是如何从``输入空间''传播到``激活空间''的?

\item \textbf{Pinsker不等式}:信息论中的Pinsker不等式建立了KL散度($\KL{\cdot}{\cdot}$)和总变差距离(TVD,$\TV{\cdot}{\cdot}$)之间的桥梁。TVD衡量了两个概率分布之间``统计差异''的绝对大小。该不等式表明:
\begin{equation}
\TV{p'}{p} \leq \sqrt{\frac{1}{2} \KL{p'}{p}}
\label{eq:pinsker}
\end{equation}

\item \textbf{应用Pinsker}:在输入分布上的攻击$\gamma$(一个$\KL{\cdot}{\cdot}$)会导致专家激活分布$p_i$变为$p'_i$。根据证明草图~\cite{moe_watermark},这个\textit{后果}(即$p_i$的\textit{统计变化})在TVD意义下被$\gamma$的\textbf{平方根}所约束:
\begin{equation}
\TV{p'_i}{p_i} \leq \sqrt{2 \gamma}
\label{eq:tvd_bound}
\end{equation}

\item \textbf{$\Dstar$的稳定性}:最后,利用Chernoff信息$\Dstar$对其基础分布$p_0, p_1$的``稳定性引理''~\cite{moe_watermark},可以证明$\Dstar$的衰减量是其基础分布TVD变化的函数。

\item \textbf{结论(机理)}:因此,检测能力$\Dstar$的衰减,其上界\textit{不是}由攻击强度$\gamma$本身决定的,而是由$\gamma$所引起的\textit{统计距离变化}($\TV{p'_i}{p_i}$)决定的。根据Pinsker不等式,这个变化被$O(\sqrt{\gamma})$牢牢限制住了。
\end{enumerate}

最终,检测能力的衰减$\Delta \Dstar$被$O(\sqrt{\gamma})$所约束。这就是\textbf{次线性衰减}。

\subsection{对比分析:线性($O(\gamma)$)vs. 次线性($O(\sqrt{\gamma})$)衰减}

现在我们可以正面回答为什么MoE范式具有机理优势。

\begin{itemize}
\item \textbf{范式A(KGW)/ 线性衰减}:$Decay \propto O(\gamma)$。
这意味着检测能力的损失与攻击强度成正比。攻击强度$\gamma$增加2倍,信号损失$\Delta Z$也增加2倍。信号是``脆弱的''。

\item \textbf{范式B(MoE)/ 次线性衰减}:$Decay \propto O(\sqrt{\gamma})$。
这意味着检测能力的损失与攻击强度的平方根成正比。
让我们看一个数值示例(如表~\ref{tab:decay_comparison}所示)。假设$\Dstar=0.1$,$C=1.5$:

\begin{itemize}
\item \textbf{攻击1}:$\gamma_1 = 0.005$(中等强度释义)。
\begin{itemize}
\item 衰减项$\approx C\sqrt{\gamma \Dstar} = 1.5 \cdot \sqrt{0.005 \cdot 0.1} \approx 0.0335$。
\item $\Dstaradv \ge 0.1 - 0.0335 = 0.0665$。(信号损失33.5\%)
\end{itemize}

\item \textbf{攻击2}:$\gamma_2 = 0.020$(\textbf{4倍}于$\gamma_1$的极强攻击)。
\begin{itemize}
\item 衰减项$\approx 1.5 \cdot \sqrt{0.020 \cdot 0.1} \approx 0.0671$。
\item $\Dstaradv \ge 0.1 - 0.0671 = 0.0329$。(信号损失67.1\%)
\end{itemize}
\end{itemize}

\textbf{关键观察}:攻击强度$\gamma$增加了\textbf{4倍}(从0.005到0.020),但信号损失$\Delta \Dstar$仅仅增加了\textbf{2倍}(从0.0335到0.0671,即$\sqrt{4}$倍)。
\end{itemize}

这就是次线性衰减的惊人之处:\textbf{攻击越强,水印信号相对于攻击强度的``韧性''就越好}。MoE范式在数学上证明了其检测能力在面对攻击时下降得更慢。

\begin{table*}[t]
\centering
\caption{线性衰减(范式A)vs. 次线性衰减(范式B)的理论影响对比}
\label{tab:decay_comparison}
\small
\begin{tabular}{|c|c|c|c|c|}
\hline
\textbf{攻击强度$\gamma$} & \textbf{范式A(KGW)$z_{\text{adv}}$} & \textbf{范式B(MoE)$\Dstaradv$(下界)} & \textbf{范式A信号保留率(\%)} & \textbf{范式B信号保留率(\%)} \\
\hline
0.00(无攻击) & 6.00 & 0.1000 & 100.0\% & 100.0\% \\
0.01 & 4.50 & 0.0526 & 75.0\% & 52.6\% \\
0.02 & 3.00 & 0.0329 & 50.0\% & 32.9\% \\
0.03 & 1.50(低于阈值) & 0.0184 & 25.0\% & 18.4\% \\
\textbf{0.04} & \textbf{0.00(信号丢失)} & \textbf{0.0051(仍可检测)} & \textbf{0.0\%} & \textbf{5.1\%} \\
0.044 & -0.60 & 0.0000(信号丢失) & 0.0\% & 0.0\% \\
\hline
\end{tabular}
\end{table*}

表~\ref{tab:decay_comparison}揭示了一个关键的细微差别。次线性衰减($O(\sqrt{\gamma})$)在$\gamma$非常小的时候,其绝对值($C\sqrt{\gamma \Dstar}$)可能大于线性衰减($k\gamma$)。然而,随着$\gamma$的增加,线性衰减会灾难性地、稳定地走向零。

如上所示,在$\gamma=0.04$时,线性衰减的范式A信号已完全丢失($z=0.0$)。而次线性衰减的范式B,虽然也遭受了重创,但其$\Dstar$仍大于0($\Dstar=0.0051$)。

$\Dstar > 0$意味着什么?根据$n^* \approx \frac{\log(1/\delta)}{\Dstar}$,它意味着水印在理论上仍然是可检测的——只是需要更多样本($n^* \approx 4.6 / 0.0051 \approx 902$个样本)。而范式A在$z=0$时,即使有无限样本也无法检测。

这就是次线性衰减的真正机理优势:它提供了鲁棒的``可检测性下限'',而线性衰减则会``灾难性地''完全失败。

\section{鲁棒性工程:从理论到实践}

MoE范式~\cite{moe_watermark}的优越性不止于理论推导;它提供了一套完整的``鲁棒性工程''框架,使这种理论优势在实践中可配置、可部署、可验证。

\subsection{安全系数($c$):连接水印强度与对手能力的桥梁}

定理~\ref{thm:robustness}不仅是一个描述性理论,更是一个\textbf{规定性}的工程工具。该研究~\cite{moe_watermark}引入了一个核心工程参数:\textbf{安全系数$c$}。

$c$的定义将水印强度$\epsilon$(即$\KL{p_1}{p_0}$约束)与\textit{预期的}对手攻击强度$\gamma$直接挂钩:

\begin{equation}
\epsilon \approx c \cdot \sqrt{\gamma} \quad (\text{或 } \Dstar \approx c^2 \gamma)
\label{eq:safety_coefficient}
\end{equation}

这是一个深刻的工程原则。范式A(KGW)只能盲目地选择一个偏置$\delta$;而范式B则允许我们根据\textit{预期的威胁模型}$\gamma$,来\textit{理论上}选择我们的防御强度$c$。

将$c$的定义代入定理~\ref{thm:robustness}的鲁棒性下界公式:

\begin{align}
\Dstaradv &\geq \Dstar - C\sqrt{\gamma \Dstar} \approx (c^2\gamma) - C\sqrt{\gamma \cdot (c^2\gamma)} \nonumber \\
\Dstaradv &\geq c^2\gamma - C c \gamma = \gamma(c^2 - Cc)
\label{eq:robustness_guarantee}
\end{align}

这个简单的公式~\cite{moe_watermark}导出了一个保证鲁棒性的\textbf{临界点}:

\begin{itemize}
\item \textbf{鲁棒性保证}:若$c > C$(根据实验标定$C \approx 1.5$--$2.0$),则$\Dstaradv > 0$。水印在理论上\textbf{保证可检测}。
\item \textbf{鲁棒性临界}:若$c = C$,则$\Dstaradv \approx 0$,水印处于失效边缘。
\item \textbf{鲁棒性失效}:若$c < C$,则$\Dstaradv$理论下界为负,鲁棒性无保证。
\end{itemize}

MoE范式提供了一条通向``可计算的鲁棒性''的清晰路径,这是KGW的z-score范式所无法企及的。

\subsection{实践中的四阶段优化框架}

理论必须落地。该研究~\cite{moe_watermark}提供了一个完整的四阶段优化框架,用于在实际部署中闭环整个理论:

\begin{enumerate}
\item \textbf{阶段1:估计$\gamma(L)$(对手标定)}:在部署模型前,首先在目标任务(如文本补全、摘要~\cite{moe_watermark})上,运行一系列释义攻击模型(如GPT-3.5, T5),测量在不同编辑距离$L$下的实际输入分布偏移$\gamma$~\cite{moe_watermark}。这为我们提供了威胁模型的基准值(例如,$\gamma \approx 0.003$--$0.01$ nats)。

\item \textbf{阶段2:标定$\Delta A(c)$(成本标定)}:测量水印的``性能成本''。通过在不同的安全系数$c$上嵌入水印,测试模型在基准任务上的精度下降$\Delta A$~\cite{moe_watermark}。这使我们得到一个成本函数$\Delta A(c)$(例如,对于7B模型,$\Delta A \approx 1.8c$百分比~\cite{moe_watermark})。

\item \textbf{阶段3:评估鲁棒性(实验验证)}:\textit{[实验验证待完成]}在实验中验证定理~\ref{thm:robustness}。通过施加阶段1中测得的$\gamma$攻击,测量在不同$c$值下的``对抗检测保留率''(即对抗后检测率 / 清洁检测率)~\cite{moe_watermark}。我们预期该保留率随$c$的增加而增加~\cite{moe_watermark}。

\item \textbf{阶段4:验证样本复杂度(理论验证)}:\textit{[实验验证待完成]}验证Chernoff-Stein定理(第~\ref{sec:chernoff}节)。比较理论所需的样本数$n^*_{\text{theory}} \approx \log(100) / \Dstar$与实验中达到99\%准确率所需的$n^*_{\text{empirical}}$~\cite{moe_watermark}。
\end{enumerate}

这个框架将纯理论(第~\ref{sec:core_derivation}节)与应用工程(选择$c$)完美地连接起来。它提供了一种可复现的、系统性的方法,用于部署一个在\textbf{已知性能预算}($\Delta A$)下,能够\textbf{可证明地}($\gamma, c > C$)抵抗\textbf{已测量威胁}($\gamma$)的水印。

\subsection{实验验证:理论的胜利}

\textit{[实验验证待完成]}该研究~\cite{moe_watermark}在LLaMA-MoE(7B, 13B, 70B)模型上进行的大量实验,以高保真度验证了上述所有理论预测:

\begin{itemize}
\item \textbf{样本复杂度验证(阶段4)}:理论预测的$n^*$与实验观测的$n^*$之间的误差始终低于15\%,通常低于10\%~\cite{moe_watermark}。这强有力地验证了Chernoff-Stein定理是描述该问题的正确模型。

\item \textbf{鲁棒性下界验证(阶段3)}:实验结果始终优于理论下界。例如,在LLaMA-7B-MoE($c=1.0$)上,面对GPT-3.5($L=5$)攻击,理论下界预测保留率为90.8\%,而实验观测值为92.4\%~\cite{moe_watermark}。这验证了定理~\ref{thm:robustness}是一个可靠且(适当)保守的鲁棒性下界。

\item \textbf{可扩展性验证(阶段2)}:实验揭示了一个关键特性——\textbf{该水印范式随模型规模的增大而变得更优}。大模型(如70B-MoE)对水印扰动的``韧性''更强,其性能下降$\Delta A(c)$更小。这意味着大模型可以``负担''得起更高的安全系数$c$(例如,7B推荐$c=1.0$,而70B推荐$c=1.8$)。
\begin{itemize}
\item 更高的$c$带来了更强的鲁棒性(70B保留率达96.2\%,高于7B的93.5\%)。
\item 更高的$c$意味着更高的$\Dstar$,因此需要\textbf{更少的检测样本}(70B仅需$n=18$,而7B需$n=37$)~\cite{moe_watermark}。
\end{itemize}
\end{itemize}

这与KGW范式~\cite{kirchenbauer2023}形成鲜明对比,后者的鲁棒性-质量权衡始终是一个难以克服的瓶颈。

\section{结论:范式转变的机理与意义}

\subsection{核心机制优势重申}

本文从信息论和假设检验的理论基础出发,严格推导了MoE专家激活水印相较于传统Token-Logit水印的机理优势。这种优势并非经验性的巧合,而是源于水印设计哲学的根本性\textbf{范式转变}。

\begin{itemize}
\item \textbf{范式A(KGW / 稠密)}:依赖于\textbf{``信号-攻击重合''}。水印信号(词元频率)与攻击方法(词元替换)在同一空间操作。这在机理上导致了$O(\gamma)$的\textbf{线性衰减}。这种设计是\textbf{脆弱的};面对中等强度的释义攻击~\cite{paraphrase},其检测信号(z-score)会灾难性地、不可逆地崩溃至零。

\item \textbf{范式B(MoE / 稀疏)}:实现了\textbf{``信号-攻击解耦''}。水印信号(专家激活分布$p_1$)被嵌入在模型的内部隐写空间~\cite{moe_watermark},与外部的攻击(输入分布偏移$\gamma$)相隔离。
\begin{itemize}
\item 这种解耦的\textbf{数学机理}体现在,攻击从输入空间到激活空间的传播受到\textbf{Pinsker不等式}的约束~\cite{moe_watermark}。
\item 这种约束\textbf{可证明地}导致了检测能力$\Dstar$的衰减呈$O(\sqrt{\gamma})$的\textbf{次线性}关系(如定理~\ref{thm:robustness}所述)~\cite{moe_watermark}。
\item 这种次线性衰减意味着水印是\textbf{鲁棒的}。它提供了可计算的保证($c > C$),确保检测能力$\Dstar$即使在强对抗下仍大于零,使得水印在理论上始终可被检测(尽管可能需要更多样本)。
\end{itemize}
\end{itemize}

\subsection{对未来水印设计的指导意义}

MoE水印框架~\cite{moe_watermark}的成功,不仅为MoE模型提供了一个强大的水印工具,更重要的是,它为设计下一代\textbf{任何}可证明鲁棒的水印系统(无论是用于文本、图像还是其他模态)提供了理论蓝图和设计原则:

\begin{enumerate}
\item \textbf{寻求解耦}:水印信号的嵌入空间\textbf{必须}与最可能的攻击向量空间相分离。不要在``像素空间''防御``JPEG压缩攻击'';也不要在``词元空间''防御``释义攻击''。

\item \textbf{采用信息论检测}:抛弃简单的频率统计(z检验),转向基于Neyman-Pearson引理~\cite{moe_watermark}的最优检测器,如似然比检验(LLR)。

\item \textbf{形式化度量}:使用真正的信息论度量(如Chernoff信息$\Dstar$~\cite{moe_watermark}或KL散度)来量化``可检测性'',而不是使用启发式的分数。

\item \textbf{形式化攻击}:将对手的能力(如释义)严格建模为可量化的参数(如分布偏移$\gamma$~\cite{moe_watermark})。

\item \textbf{证明边界}:推导连接攻击强度($\gamma$)和检测能力衰减($\Delta \Dstar$)的数学边界(如定理~\ref{thm:robustness}~\cite{moe_watermark})。
\end{enumerate}

MoE水印框架~\cite{moe_watermark}是第一个在LLM领域完整实现了这五大原则的实用系统,它将水印从``启发式技巧''提升到了``可计算的工程科学''的高度,为解决``对抗性擦洗''这一核心难题提供了第一个理论完备的答案。

\section*{致谢}

作者感谢信息论、统计假设检验和水印技术领域的前期研究奠定的理论基础。实验验证和详细证明标记为待完成工作。

\bibliographystyle{abbrv}
\begin{thebibliography}{99}

\bibitem{kirchenbauer2023}
J. Kirchenbauer et al., ``A Watermark for Large Language Models,'' in \textit{Proc. ICML}, 2023.

\bibitem{moe_watermark}
[核心研究参考文献], ``MoE水印框架,'' \textit{待引用}, 2024.

\bibitem{moe_arch}
S. Shazeer et al., ``Outrageously Large Neural Networks: The Sparsely-Gated Mixture-of-Experts Layer,'' \textit{arXiv preprint}, 2017.

\bibitem{scrubbing}
[擦洗攻击参考文献], ``水印的对抗性擦洗攻击,'' \textit{待引用}, 2023.

\bibitem{token_fragile}
[词元脆弱性参考文献], ``基于词元的水印的脆弱性,'' \textit{待引用}, 2023.

\bibitem{paraphrase}
[释义攻击参考文献], ``水印文本的释义攻击,'' \textit{待引用}, 2023.

\bibitem{lexical_edit}
[词汇编辑参考文献], ``词汇编辑攻击,'' \textit{待引用}, 2023.

\bibitem{cross_lang}
[跨语言参考文献], ``跨语言回译攻击,'' \textit{待引用}, 2023.

\end{thebibliography}

\end{document}

