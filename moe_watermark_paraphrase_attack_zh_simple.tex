\documentclass[11pt,a4paper]{ctexart}

\usepackage{amsmath}
\usepackage{amssymb}
\usepackage{amsthm}
\usepackage{geometry}
\geometry{margin=2.5cm}

% Theorem environments
\newtheorem{theorem}{定理}
\newtheorem{definition}{定义}

% Custom commands
\newcommand{\KL}[2]{D_{\text{KL}}(#1 \| #2)}
\newcommand{\TV}[2]{\| #1 - #2 \|_{\text{TV}}}
\newcommand{\Dstar}{D^*}
\newcommand{\Dstaradv}{D^*_{\text{adv}}}

\title{MoE水印对抗释义攻击的核心机理优势\\简化版}

\author{yunhao}

\date{}

\begin{document}

\maketitle

\section{核心问题}

传统词元-逻辑值(Token-Logit)水印在面对释义攻击时存在根本性脆弱性。本文从信息论角度证明,基于MoE专家激活的水印范式具有机理上的根本优势。

\section{两种范式的核心差异}

\subsection{范式A:稠密模型水印(KGW)}

\textbf{信号载体}:词元概率分布(词汇表空间)

\textbf{攻击向量}:词元替换(词汇表空间)

\textbf{核心问题}:信号与攻击在同一空间,导致\textbf{信号-攻击重合}

\textbf{衰减机理}:线性衰减
\begin{equation}
\Delta Z \propto \gamma
\end{equation}
其中$\gamma$为攻击强度(被替换词元比例)。

\subsection{范式B:MoE水印}

\textbf{信号载体}:专家激活分布$p(e|x)$(内部激活空间)

\textbf{攻击向量}:词元替换$x \rightarrow x'$(外部词元空间)

\textbf{核心优势}:信号与攻击在不同空间,实现\textbf{信号-攻击解耦}

\textbf{衰减机理}:次线性衰减
\begin{equation}
\Delta \Dstar \propto O(\sqrt{\gamma})
\end{equation}

\section{机理推导}

\subsection{攻击建模}

将释义攻击$x \rightarrow x'$建模为输入分布偏移:
\begin{equation}
\gamma = \KL{D(X')}{D(X)}
\end{equation}

\subsection{核心定理}

\begin{theorem}[对抗鲁棒性]
在强度为$\gamma$的释义攻击下,MoE水印的检测能力(Chernoff信息)满足:
\begin{equation}
\Dstaradv \geq \Dstar(p_0, p_1) - C\sqrt{\gamma \cdot \Dstar(p_0, p_1)} - O(\gamma)
\label{eq:robustness_bound}
\end{equation}
其中$C \approx 1.5$--$2.0$为常数。
\end{theorem}

\textit{证明思路}:通过Pinsker不等式,输入空间的KL散度$\gamma$传播到激活空间时,被约束为$\sqrt{\gamma}$量级的总变差距离,从而检测能力衰减为次线性。

\subsection{为什么是$\sqrt{\gamma}$?}

\textbf{关键步骤}:

1. \textbf{空间解耦}:攻击$\gamma$在输入空间,信号$\Dstar$在激活空间

2. \textbf{Pinsker不等式}:
\begin{equation}
\TV{p'}{p} \leq \sqrt{\frac{1}{2} \KL{p'}{p}}
\end{equation}

3. \textbf{传播约束}:输入分布的攻击$\gamma$导致激活分布变化:
\begin{equation}
\TV{p'_i}{p_i} \leq \sqrt{2 \gamma}
\end{equation}

4. \textbf{结论}:检测能力衰减$\Delta \Dstar$被$O(\sqrt{\gamma})$约束

\section{对比分析}

\subsection{数值示例}

假设初始$\Dstar=0.1$,$C=1.5$:

\begin{itemize}
\item \textbf{攻击1}:$\gamma_1 = 0.005$
  \begin{itemize}
  \item 衰减项:$1.5 \cdot \sqrt{0.005 \cdot 0.1} \approx 0.0335$
  \item 剩余$\Dstaradv \ge 0.0665$(损失33.5\%)
  \end{itemize}

\item \textbf{攻击2}:$\gamma_2 = 0.020$(4倍于攻击1)
  \begin{itemize}
  \item 衰减项:$1.5 \cdot \sqrt{0.020 \cdot 0.1} \approx 0.0671$
  \item 剩余$\Dstaradv \ge 0.0329$(损失67.1\%)
  \end{itemize}
\end{itemize}

\textbf{关键观察}:攻击强度增加4倍,信号损失仅增加2倍($\sqrt{4}=2$)。

\subsection{线性 vs 次线性衰减}

\begin{center}
\begin{tabular}{|c|c|c|}
\hline
\textbf{攻击强度$\gamma$} & \textbf{范式A(线性)} & \textbf{范式B(次线性)} \\
\hline
0.01 & 信号保留75\% & 信号保留52.6\% \\
0.02 & 信号保留50\% & 信号保留32.9\% \\
0.04 & \textbf{信号完全丢失} & \textbf{仍可检测(5.1\%)} \\
\hline
\end{tabular}
\end{center}

\textbf{核心优势}:次线性衰减在强攻击下仍保持可检测性,而线性衰减会灾难性失效。

\section{工程框架}

\subsection{安全系数$c$}

将水印强度$\epsilon$与预期攻击强度$\gamma$关联:
\begin{equation}
\epsilon \approx c \cdot \sqrt{\gamma} \quad \text{或} \quad \Dstar \approx c^2 \gamma
\end{equation}

\textbf{鲁棒性保证}:若$c > C$(约1.5--2.0),则$\Dstaradv > 0$,水印理论上保证可检测。

\section{核心结论}

\begin{enumerate}
\item \textbf{信号-攻击解耦}是MoE水印鲁棒性的根本原因

\item \textbf{次线性衰减}($O(\sqrt{\gamma})$)vs \textbf{线性衰减}($O(\gamma)$)是机理上的根本差异

\item 通过Pinsker不等式,输入空间的攻击被约束为激活空间的$\sqrt{\gamma}$量级变化

\item 次线性衰减提供了\textbf{可检测性下限},而线性衰减会完全失效
\end{enumerate}

\section*{备注}

\begin{itemize}
\item 实验验证:待完成
\item 详细证明:待后续补充
\item 本文仅展示核心理论机理
\end{itemize}

\end{document}

